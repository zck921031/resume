% !TEX TS-program = xelatex
% !TEX encoding = UTF-8 Unicode
% !Mode:: "TeX:UTF-8"

\documentclass{resume}
\usepackage{zh_CN-Adobefonts_external} % Simplified Chinese Support using external fonts (./fonts/zh_CN-Adobe/)
%\usepackage{zh_CN-Adobefonts_internal} % Simplified Chinese Support using system fonts
\usepackage{linespacing_fix} % disable extra space before next section
\usepackage{cite}
\usepackage[colorlinks, linkcolor=black,anchorcolor=black,citecolor=black]{hyperref}

\begin{document}
\pagenumbering{gobble} % suppress displaying page number

\name{朱铖恺}
% {E-mail}{mobilephone}{homepage}
% be careful of _ in emaill address
%\contactInfo{zck921031@qq.com}{}{}
%\centerline{\sffamily\large{\faEnvelope\ {zck921031@qq.com} \textperiodcentered\ \faPhone\ {(+86) 183-2248-0965} \textperiodcentered\ \faBirthdayCake\ {1992.10.31} \textperiodcentered\ \faHome\ {福建三明}}
%	\vspace{1.5ex} }
\centerline{\sffamily\large{\faEnvelope\ {zck921031@qq.com} \ \faWeixin\ zck921031 \ \faPhone\ {(+852) 6764-2081} \ \faBirthdayCake\ {1992.10.31} \ \faHome\ {福建三明}} \vspace{1.0ex}}
% {E-mail}{mobilephone}
% keep the last empty braces!
%\contactInfo{xxx@yuanbin.me}{(+86) 131-221-87xxx}{}
 
\section{\faGraduationCap\  教育背景}
\datedsubsection{\textbf{天津大学}\ 天津}{2015 -- 至今}
\textit{在读工学硕士}\ 计算机科学与技术, 预计 2018 年 1 月毕业
\datedsubsection{\textbf{东南大学}\ 江苏, 南京}{2011 -- 2015}
\textit{工学学士}\ 测控技术与仪器

\section{\faUsers\ 实习经历}
\datedsubsection{\textbf{香港科技大学}\ 香港}{2017年4月 -- 至今}
\role{研究助理 (Research Assistant)}{\ 计算机科学与工程系}
于2017年4月加入\textsl{基于异构大数据的智能交通应用}项目,主要负责将计算机视觉和深度学习算法应用于监控视频分析。
我们团队基于密度图估计的策略,实现了密集人群场景下的人数估计。
目前我们致力于密集人群的流向估计和异常事件检测,以避免潜在危险事故的发生。

\section{\faSearch\ 科研与项目}
\datedsubsection{\textbf{基于深度学习的模糊图像分类}}{2016年8月 -- 至今}
\role{研究深度网络的训练方法}{}
\begin{onehalfspacing}
	此项目致力于将深度卷积神经网络用于模糊图像分类,例如人脸表情的强度预测。我们发现常用的损失函数在这个问题中表现的不好。因此,我们在这个工作中提出了一个新的模糊粗糙损失函数,通过最小化模糊分类不确定度训练深度网络。由于不同的人对同样的情感表达各不相同,而我们无法收集所有人的数据,因此我们进一步采用了多任务对抗损失函数进行迁移学习。
\end{onehalfspacing}
\datedsubsection{\textbf{东南大学在线评测系统}}{2013年3月 -- 2014年6月}
\role{SeuOJ@CentOS}{与软件学院合作}
\begin{onehalfspacing}
	基于开源在线评测系统\href{https://github.com/zhblue/hustoj}{HustOJ}项目,二次开发并部署于CentOS服务器。我的工作是部署和维护后端判题守护程序。在项目过程中我发现HustOJ项目的判题程序在CentOS 6下存在线程阻塞问题。通过调整降权顺序,修复了该问题,提高了服务器的并发判题效率。可以通过\href{http://acm.seu.edu.cn/oj/}{http://acm.seu.edu.cn/oj}访问。
	%\begin{itemize}
	%  \item 负责后端判题程序的部署和维护。
	%  \item 发现并修复了判题程序降权过早导致的线程阻塞问题,提高了判题效率。
	%  \item 成功举办2013、2014年东南大学程序设计竞赛,5小时无阻塞评测多达2883份代码。
	%\end{itemize}
\end{onehalfspacing}

\section{\faHeartO\ 获奖情况}
\begin{itemize}[parsep=0.5ex]
	\item \datedline{ACM国际大学生程序设计竞赛(ACM-ICPC)亚洲区域赛 \textit{北京赛区金奖,杭州赛区金奖,广州赛区银奖,长沙赛区银奖}}{2013 年 10月 -- 2014 年 11 月}
	\item \datedline{CCF全国青年大数据创新大赛·中文地址魔方大赛 \textit{冠军}}{2016 年 1 月}
	\item \datedline{“掌赢杯”南京市大学生程序设计大赛 \textit{亚军}}{2014 年 5 月}
	\item \datedline{“华为杯”苏鲁高校大学生程序设计大赛 \textit{亚军}}{2014 年 5 月}
%	\item \datedline{华为“编码合伙人”大赛 \textit{季军}}{2014 年 7 月}
	\item \datedline{一等奖学金、华为奖学金、正保教育奖学金、星网锐捷奖学金、张志伟奖学金}{2012年 -- 2016年}
\end{itemize}


% Reference Test
%\datedsubsection{\textbf{Paper Title\cite{zaharia2012resilient}}}{May. 2015}
%An xxx optimized for xxx\cite{verma2015large}
%\begin{itemize}
%  \item main contribution
%\end{itemize}

\section{\faCogs\ IT 技能}
% increase linespacing [parsep=0.5ex]
\begin{itemize}[parsep=0.5ex]
  \item 研究领域: 深度学习(with \href{http://caffe.berkeleyvision.org/}{caffe} and \href{http://deeplearning.net/software/theano/}{theano}), 计算机视觉(with \href{http://opencv.org/}{OpenCV})
  \item 编程语言: C++, Java, Python
%  \item 平台: 多年Linux使用经验,熟悉命令与配置,能通过编写脚本提高效率。
  \item 语言: 英语(CET6), 能流畅阅读英文资料,在英文邮件列表、Github 等社区正常交流。
  \item 能够使用 LaTeX 及 Markdown 写作,具备良好的交流能力。
  \item \faGithub Github \href{https://github.com/zck921031}{https://github.com/zck921031}
%  \item Topcoder(Top Rating: 1819):\  \href{https://www.topcoder.com/members/zck921031/}{https://www.topcoder.com/members/zck921031/}
\end{itemize}

%% Reference
%\newpage
%\bibliographystyle{IEEETran}
%\bibliography{mycite}
\end{document}
